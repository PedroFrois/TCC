\chapter[Trabalhos Relacionados]{Trabalhos Relacionados}
\label{cap:trabalhosRelacionados}

Eke (2019) propõe um SDK (Software Development Kit) a ser utilizado pela empresa Popit Oy \cite{triviaPopit}. No plano secundário, tem por objetivos listar princípios do estado da arte para desenvolvimento de SDKs e validar as seguintes hipóteses: se a arquitetura proposta atende aos requisitos não funcionais do produto e se o \textit{framework} Kotlin Multiplatfom Mobile já está maduro o suficiente para uso empresarial. 

Para responder todas as questões levantadas e atingir os objetivos propostos a metodologia do trabalho foi dividida em três etapas principais: 

\begin{enumerate}
    \item pesquisa das melhores práticas de desenvolvimento de SDKs;
    \item desenvolvimento da SDK;
    \item a validação de seu trabalho.
\end{enumerate}

Não obstante ter atingido o objetivo principal do trabalho, consistente no desenvolvimento de um SDK que atendesse a todos os requisitos não funcionais do projeto, os resultados obtidos sugerem que o \textit{framework} Kotlin Multiplatform Mobile (KMM) não é recomendado para uso empresarial.

Contudo, em sua conclusão, Eke (2019) sugeriu que esse cenário provavelmente mudaria ao final do ano de 2019 com novas atualizações do Kotlin, tendo em vista que o maior motivo dos desafios do desenvolvimento utilizando esse \textit{framework} foi sua prematuridade. Ademais, acrescentou que, com a resolução desses pequenos desafios, o KMM se tornará o padrão para desenvolvimento multiplataforma para compartilhamento de regras de negócio \cite{multiplataformaMedicalKit}.

%===================================================================================================================
Cheon (2019) propõe uma forma de desenvolvimento para aplicações multiplataforma que consiste na criação de duas aplicações específicas para cada plataforma e uma biblioteca, ou SDK, para compartilhamento de código comum. 

Para avaliar a proposta, foi realizado um estudo empírico em que o autor desenvolveu uma aplicação multiplataforma Java e Android e relatou sua opinião sobre os procedimentos aventados. Foi concluído que a abordagem mais prática para o desenvolvimento seria:

\begin{itemize}
    \item em caso de funcionalidades mais básicas e bem conhecidas: primeiro desenvolver para o SDK comum e despois para as aplicações específicas;
    
    \item em caso de funcionalidades complexas e não tão bem conhecidas: é melhor desenvolver para uma aplicação específica e, após solucionado o problema, generalizar a solução para exportá-la para o SDK comum e depois utilizar a solução genérica nas duas aplicações específicas.
\end{itemize}
Outros resultados e observações do autor são que o código dividido conforme a abordagem multiplataforma fica mais verboso, porém, 37\%~40\% do código foi reutilizado fazendo com que a manutenção fosse bem mais fácil e eficaz, podendo assim, aumentar a qualidade da aplicação.
  
%==================================================================================================================
O último trabalho relacionado faz uma análise comparativa entre o desenvolvimento de aplicações nativas iOS e Android e aplicações utilizando o KMM. O objetivo do trabalho é responder as seguintes questões: 

\begin{enumerate}
    \item Quais são os efeitos do uso do KMM na percepção do usuário?
    \item Quais são os efeitos do uso do KMM analisando a aplicação instalada?
    \item Quais são os efeitos do uso do KMM quanto à produtividade do desenvolvedor?
\end{enumerate}

Com o foco na resolução dessas questões, Evert (2019) definiu as métricas a serem utilizadas: tempo de inicialização da aplicação; tamanho em memória da aplicação; e, para análise de produtividade, a quantidade de linhas de código e o tempo de \textit{build} do projeto. Além disso definiu um projeto com utilidade real para sua avaliação ser mais fiel ao mercado. Seguindo o proposto, a autora atingiu seu objetivo, cujos resultados serão expostos de maneira resumida, para não fugir do escopo deste trabalho. Os resultados encontrados são descritos a seguir:

\begin{itemize}
    \item Tamanho da aplicação: a aplicação multiplataforma é um pouco maior do que a nativa para os dois sistemas operacionais. Mais especificamente, 13\% e 18\% maior no Android e iOS, respectivamente.
    
    \item O tempo de inicialização da aplicação foi um pouco maior em Android ao se utilizar a abordagem multiplataforma - houve um aumento de 587,4ms na média para 675,6ms. Já em iOS, não se observou alteração no tempo de inicialização
    
    \item Linhas de código: a soma das linhas de código dos projetos nativos Android e iOS chegaram a 494, sendo que na abordagem utilizando KMM o total de linhas de código foram 383. Isso corresponde a uma redução de 18\%.
    
    \item O tempo de \textit{build}: tanto em Android quanto em iOS houve acréscimo no tempo de \textit{build}.
\end{itemize}
Na conclusão, Evert (2019) comenta que os ganhos de produtividade obtidos com a redução de linhas de código podem ser superados pela perda de produtividade ocasionada por tempos de \textit{build} maiores. Além disso, se o tamanho da aplicação e o tempo de inicialização forem essenciais para o projeto, KMM pode não ser a melhor opção a ser utilizada \cite{multiplataformaCross-plataform-diva}.

%==================================================================================================================

Tabian (2021) realizou um trabalho prático a fim de responder a pergunta se \textit{Kotlin Multiplataform Mobile (KMM)} está pronto para ser utilizado em produção. Para atingir seu objetivo ele se propôs desenvolver o mesmo aplicativo tanto para Android quanto para iOS compartilhando certa parte do código do fonte utilizando o KMM. 

Além disso seu trabalho listou e deu seu parecer sobre todas as tecnologias que utilizou para o desenvolvimento  ao final do projeto. Segundo ele algumas bibliotecas utilizadas pelo KMM ainda não possuem uma base de usuários consolidada, então ao enfrentar problemas com elas o desenvolvedor terá que, muito provavelmente, resolvê-los por conta própria sem a ajuda de outros.

Em sua conclusão, apesar de ter enfrentado alguns problemas de configurações iniciais, Tabian (2021) afirma que KMM pode ser utilizado para desenvolvimento de aplicações que visam entrar em produção, dado que ele conseguiu desenvolver uma aplicação Android e iOS que utilizam KMM e estão disponíveis para \textit{download} nas lojas oficiais da Google e Apple.

%==================================================================================================================

ADICIONAR PARÁGRAFO COMENTANDO SOBRE VÍDEO.


%==================================================================================================================
O trabalho aqui desenvolvido, apesar de ter sido idealizado sem o conhecimento da já mencionada dissertação de Eke (2019), se assemelha bastante a ela, com algumas ressalvas. A arquitetura aqui proposta tem um embasamento teórico superior e visa atender domínios genéricos em que a solução KMM possa ser utilizada. Ademais, o trabalho de Eke (2019) será utilizado como referência para melhor análise do \textit{framework} KMM com a finalidade de verificar se a arquitetura pode ter uma maior dependência do KMM a fim de reutilizar mais partes de código ou não.

Já o trabalho de Cheon (2019) demonstrou uma divisão de arquitetura baseada em componentes dependentes ou independentes da plataforma, porém, não a classificou quanto à termos comuns utilizados para se referenciar a camadas em arquitetura. Com isso, a arquitetura de seu trabalho ficou muito abrangente e de difícil reprodução para outros trabalhos. A arquitetura a ser proposta na Seção \ref{cap:novaArquitetura} será melhor detalhada e de mais fácil aplicação a qualquer domínio de problema.

Quanto ao trabalho de Evert (2019), apesar de não descrever nenhum tipo de arquitetura, foi utilizado para conhecimento sobre o \textit{framework} KMM que, não obstante já ter sido lançado oficialmente, ainda possui alguns desafios, como mencionados por Eke (2019).


ADICIONAR PARÁGRAFO COMENTANDO SOBRE VÍDEO.

%==================================================================================================================

Em suma, os trabalhos relacionados têm um teor mais prático, incluindo a implementação de um programa funcional. Todavia, deixam de lado uma boa justificativa e descrição teórica para motivar o uso da arquitetura escolhida. O presente trabalho diferencia-se dos outros uma vez que só irá abordar o escopo da arquitetura, a qual terá seu desenvolvimento fundamentado em literaturas já consolidadas e conhecidas pelo meio acadêmico e pelo mercado.


ADICIONAR comparação COMENTANDO SOBRE VÍDEO.