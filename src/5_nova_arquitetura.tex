\chapter[Descrição da Arquitetura Proposta]{Descrição da Arquitetura Proposta}
\label{cap:novaArquitetura}

A proposta da nova arquitetura tomou como base principalmente outras duas, a \textit{Model-View-ViewModel} (MVVM) e a \textit{Clean Architecture} \cite{arquiMVVMGoogle, arquiCleanArch}. Essas duas arquiteturas serão descritas nas seções \ref{sec:mvvm} e \ref{sec:clean}, entretanto para um melhor entendimento é recomendado a leitura referenciada de cada uma.

\section{Arquitetura \textit{Model-View-Viewmodel}}
\label{sec:mvvm}

Trata-se de uma arquitetura criada por Ken Cooper e Ted Peters, sendo atualmente recomendada pela Google para o desenvolvimento de aplicações nativas em Android. Anteriormente, até 2010, a arquitetura recomendada era a \textit{Model-View-Controller}, mas, por dificuldades de manutenção e testes, a Google passou a recomendar a MVVM ou \textit{Model-View-Presenter}. Por meio do artigo "\textit{A comparison of Android Native App Architecture MVC, MVP and MVVM}", foi realizada a comparação entre as 03 (três) arquiteturas, ocasião na qual conclui-se que tanto a MVVM quanto MVP possuem ganhos significativos quando comparadas com a MVC. Já entre MVVM e MVP, não houve uma resposta clara acerca de qual seria a melhor, mas foi concluído que MVVM é mais facilmente testável, enquanto a MVP tinha uma manutenibilidade melhor \cite{arquiMVVM, arquiMVVMGoogle}.

Dado o exposto, tem-se que a escolha entre MVVM e MVP se baseiam no domínio do problema e no consenso entre os desenvolvedores a respeito de qual característica é a mais relevante para seu projeto. Todavia, sistemas em que testes são realizados mais facilmente são melhor testados, logo, possuem menos erros críticos em produção, de forma que as manutenções que serão realizadas para correção de tais erros dispõem de um tempo maior. Deste modo, a arquitetura MVVM foi a escolhida para ser utilizada como uma das inspirações dessa nova arquitetura que será apresentada na subseção \ref{sec:description}.

A representação de alto nível da arquitetura pode ser observada na Figura \ref{fig:mvvm}. A partir dela, é possível compreender que a \textit{View} e a \textit{ViewModel}, representados, respectivamente, na figura por \textit{Activity/Fragment} e \textit{ViewModel}, são responsáveis pela visualização e lógica de visualização do software. Já o agrupamento do \textit{Repository}, \textit{Model} e \textit{Remote Data Source} gerenciam as partes de modelo de dados, regras de negócio e requisição de dados local e remoto. Esse agrupamento foi abstraído para o nome \textit{Model}, que está presente no nome da arquitetura. 


\begin{figure}[h]
    \caption{Arquitetura Model-View-Viewmodel}
    \centering
    \includegraphics[width=.85\textwidth]{img/mvvm.png}
    \fonte{\cite{arquiMVVMGoogle}}
    \label{fig:mvvm}
\end{figure}


\section{Arquitetura Limpa - \textit{Clean Architecture}}
\label{sec:clean}

A \textit{Clean Architecture} ou Arquitetura Limpa foi desenvolvida por Robert Martin \cite{arquiCleanArch} tem como foco a \textit{Separation of Concerns}, ou Separação de Conceitos, fazendo com que todos os conceitos SOLID explicados anteriormente sejam seguidos. Ademais, promove a "[...] implementação de sistemas que favorecem reusabilidade de código, coesão, independência de tecnologia e testabilidade." \cite{arquiEngSoftModerna}.

Essa arquitetura utiliza-se de outras arquiteturas conhecidas como: Arquitetura Hexagonal \cite{freeman2009growing}, DCI \cite{reenskaug2009dci} e BCE \cite{jacobson1993object},as quais abordam o conceito de divisão de camadas a fim de atingirem os mesmos objetivos. São arquiteturas: 

\begin{itemize}
    \item Independentes de \textit{frameworks};
    \item Testáveis;
    \item Independentes da Interface de Usuário;
    \item Independente do Banco de dados;
    \item Independente de qualquer fator externo.
\end{itemize}

Segundo Martin, a Arquitetura Limpa surgiu da "integração de todas essas arquiteturas em uma única ideia executável" \cite{arquiCleanArch}. Com isso, é de fácil compreensão o motivo pelo o qual a Arquitetura Limpa serviu de inspiração para a que será exposta na subseção \ref{sec:description}.

Para entender a Arquitetura Limpa, pode-se observar a Figura \ref{fig:clean}. A ideia geral por trás da arquitetura é ter camadas centrais que são raramente alteradas e desconhecem as camadas externas a elas. Quanto mais externa, mais concreta e ligada à plataforma de uso é a camada. Exemplificando: as entidades são representações de estruturas do domínio de seu software, então, independente de qualquer linguagem, programa, plataforma ou qualquer decisão nesse sentido, as entidades de seu programa não serão alteradas. Essa dinâmica é representada na Figura \ref{fig:cleanCone}.

\begin{figure}[h]
    \caption{Clean Architecture - Arquitetura Limpa}
    \centering
    \includegraphics[width=1\textwidth]{img/cleanArch.png}
    \fonte{\cite{arquiCleanArch}}
    \label{fig:clean}
\end{figure}

\begin{figure}[h]
    \caption{Clean Architecture Cone - Cone da Arquitetura Limpa}
    \centering
    \includegraphics[width=1\textwidth]{img/cleanArchCone.png}
    \fonte{\cite{arquiCleanStack}}
    \label{fig:cleanCone}
\end{figure}

\newpage


\section{Fundamentos de decisão de projeto}
\label{sec:Fundamentos}

Nesta seção serão elencados os principais fundamentos para as decisões de projeto que foram tomadas para o desenvolvimento da arquitetura proposta.

Conforme mencionado na seção \ref{sec:mvvm}, a arquitetura MVVM é facilmente testável e a atualmente recomendada pela google, porém possui a falha de não ter uma manutenibilidade tão boa quanto à MVP \cite{arquiMVVM}.

Com intuito de ter uma arquitetura facilmente testável, porém também com fácil manutenção, objetiva-se fazer uma adaptação da MVVM ao mesclar com a arquitetura Limpa. Ao realizar essa mescla, a Separação de Conceitos da arquitetura Limpa ajudará com a aplicação de todos os princípios SOLID \cite{arquiCleanArch}.

Além dos princípios SOLID, outros pontos relevantes para a fundamentação da arquitetura são: 

\begin{itemize}
    \item padrão de projeto \textit{Observer}: esse padrão de projeto possibilita a comunicação intermodular reduzindo o acoplamento uma vez que somente o módulo observador precisa conhecer o módulo observado.
    \item telas sem textos ou valores padrão fixos: com esses dados sendo injetados por um módulo específico ou sendo enviados pelo \textit{back-end}, a manutenção torna-se mais simples e dinâmica para qualquer plataforma que consuma esse módulo. A manutenção será feita somente em um local ao invés de dois (no caso de plataformas Android e iOS).
    \item padronizar a criação de novas telas: facilita a manutenção uma vez que, o fluxo de criação e todas as classes referentes a cada funcionalidade, possuem o mesmo padrão de desenvolvimento e estrutural.
\end{itemize}


\newpage

\section{Descrição da arquitetura proposta}
\label{sec:description}
A arquitetura proposta será divida em dois grandes módulos, a partir de uma abstração genérica da arquitetura MVVM, já utilizada e verificada pelo uso público. Neste modelo de inspiração, o primeiro módulo é o \textit{Model}, que representa toda a parte de entidades e regras de negócio, enquanto o segundo é a junção da \textit{View} e \textit{ViewModel} de forma que exista um módulo específico e responsável pela visualização e regras de apresentação ao usuário.

Já a inspiração na Arquitetura Limpa se deu para detalhar mais a fundo as camadas internas e suas responsabilidades, a fim de garantir uma definição melhor e mais prática do que será contido em cada módulo para facilitar o entendimento e reprodução da arquitetura.

De forma genérica e, por enquanto, abstrata, os dois módulos da arquitetura proposta, conforme se visualiza na Figura \ref{fig:propostaArquitetura}, podem ser definidos em:

\begin{enumerate}
    \item \textit{Domain}: análago ao Model da arquitetura MVVM e utilizando-se dos princípios da Arquitetura Limpa, ele é o módulo central da arquitetura, de forma que deverá ser totalmente independente da plataforma de implementação. Com isso, ele poderá ser completamente intercambiável entre Android e iOS, de modo que consistirá na representação da multiplataforma nessa arquitetura.
    
    \item \textit{Presentation}: é o modulo responsável pela apresentação e regras de interação com o usuário. Ele é totalmente dependente da plataforma que se está utilizando de modo que, para um desenvolvimento multiplataforma no contexto Android e iOS, tudo que for implementado dentro desse módulo para Android deverá ser reimplementado para iOS.
\end{enumerate}

\begin{figure}[h]
    \caption{Proposta Geral da Arquitetura}
    \centering
    \includegraphics[width=.6\textwidth]{img/Arquitetura Basica - Interacao.png}
    \fonte{Própria}
    \label{fig:propostaArquitetura}
\end{figure}
