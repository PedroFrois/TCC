\chapter[Metodologia]{Metodologia}
\label{cap:metodologia}

A Figura \ref{fig:metodologia} ilustra as etapas da metodologia para o desenvolvimento do trabalho proposto dividida em 4 atividades - pesquisa (1); avaliação (3); descrição da avaliação (5); avaliação final (7) - que resultam em 4 artefatos - lista de boas práticas para arquitetura de software (2); base para descrição da arquitetura (4); proposta de arquitetura (6); e a arquitetura final com observações da avaliação (8).

\begin{figure}[h]
    \caption{Processo de trabalho}
    \centering
    \includegraphics[width=1\textwidth]{img/fluxograma-metodologia.png}
    \fonte{Própria}
    \label{fig:metodologia}
\end{figure}
Agora as etapas serão detalhadas e explicadas:

\begin{enumerate}
    \item Pesquisa
    
    Para a pesquisa com foco na procura de arquitetura de software foram utilizadas referências recomendadas pela orientadora e pelo coorientador desse trabalho além de uma procura ampla utilizando a ferramenta \textit{google scholar}. As palavras chave utilizadas foram: "Multiplatform Android Architecture", "Multiplatform Software Architecture", "Kotlin Multiplataform Mobile" e "Mobile Architecture".
    
    \item Lista de boas práticas
    
    É o produto bruto da pesquisa feita em que serão listadas as melhores práticas para desenvolvimento de uma arquitetura junto à pontos que influenciam na qualidade da arquitetura.
    
    \item Avaliação
    
    A avaliação que irá selecionar os pontos relevantes para o domínio do trabalho de forma e gerar a lista base a ser utilizada neste trablho.
    
    \item Base para descrição da arquitetura
    
    São as informações base gerada pela etapa anterior. Reune as melhores práticas para desenvolvimento de uma arquitetura e os pontos que influenciam em sua qualidade sendo ambos relevantes para o domínio do trabalho. Essa base será utilizada para guiar a descrição da arquitetura proposta por essa dissertação.
    
    \item Descrição da arquitetura
    
    Utilizando-se do artefato obtido anteriormente, será descrita a arquitetura a fim de atingir o objetivo desse trabalho.
    
    \item Proposta de arquitetura
    
    O artefato gerado pela atividade anterior.
    
    \item Avaliação
    
    Avaliação da proposta de arquitetura a fim de elucidar pontos de melhora, pontos que foram atendidos e, principalmente, se a arquitetura atinge o objetivo do trabalho.
    
    \item Arquitetura final com observações da avaliação
    
    Artefato final do trabalho contendo a proposta da arquitetura em conjunto com a avaliação da mesma.

\end{enumerate}
