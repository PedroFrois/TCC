\chapter[Metodologia]{Metodologia}
\label{cap:metodologia}

A Figura \ref{fig:metodologia} ilustra as etapas da metodologia para o desenvolvimento do trabalho proposto. Ela é dividida em 04 (quatro) atividades: pesquisa (1); avaliação (3); descrição da arquitetura (5); avaliação da arquitetura proposta (7). Tais atividades resultam em 04 (quatro) artefatos: lista de características para uma boa arquitetura (2); base para descrição da arquitetura (4); proposta de arquitetura (6); e a arquitetura final com observações da avaliação (8).

\begin{figure}[h]
    \caption{Processo de trabalho}
    \centering
    \includegraphics[width=.52\textwidth]{img/fluxograma-metodologia.png}
    \fonte{Própria}
    \label{fig:metodologia}
\end{figure}

Detalhamento e explicação das etapas:

\begin{enumerate}
    \item Pesquisa
    
    Para a pesquisa com foco na procura de arquitetura de software foram utilizadas referências recomendadas pela orientadora e pelo coorientador deste trabalho, além de uma pesquisa ampla utilizando a ferramenta Google Scholar. 
    
    As palavras chave utilizadas foram: \textit{"Multiplatform Android Architecture"}, \textit{"Multiplatform Software Architecture"}, \textit{"Kotlin Multiplataform Mobile"} e \textit{"Mobile Architecture"}.
    
    \item Lista de boas práticas
    
    É o produto bruto da pesquisa feita, em que serão listadas as melhores práticas para desenvolvimento de uma arquitetura junto aos pontos que influenciam em sua qualidade.
    
    \item Avaliação
    
    A avaliação irá selecionar as questões relevantes para o domínio do trabalho de forma, bem como gerar a lista base a ser utilizada neste trabalho.
    
    \item Base para descrição da arquitetura
    
    São as informações base geradas pela etapa anterior. Reúne as melhores práticas para desenvolvimento de uma arquitetura e as questões que influenciam em sua qualidade, sendo ambos relevantes para o domínio do trabalho. Referida base será utilizada para guiar a descrição da arquitetura proposta por essa dissertação.
    
    \item Descrição da arquitetura
    
    Utilizando-se do artefato obtido anteriormente, será descrita a arquitetura a fim de atingir o objetivo do trabalho.
    
    \item Proposta de arquitetura
    
    Consiste no artefato gerado pela atividade anterior.
    
    \item Avaliação
    
    Avaliação da proposta de arquitetura a fim de elucidar pontos de melhora, pontos que foram atendidos e, principalmente, se a arquitetura atinge o objetivo do trabalho.
    
    \item Arquitetura final com observações da avaliação
    
    Artefato final do trabalho contendo a proposta da arquitetura em conjunto com a avaliação da mesma.

\end{enumerate}
