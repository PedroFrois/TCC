\chapter[Introdução]{Introdução}
\label{cap:introducao}

Devido à constante introdução no mercado de novas tecnologias, \textit{frameworks} e áreas de utilização, as aplicações móveis possuem um caráter dinâmico de desenvolvimento \cite{multiplataformaCross-plataform-diva}. Neste  universo, dois sistemas operacionais são dominantes no cenário de aplicações móveis, Android e iOS \cite{multiplataformaCross-plataform-diva}, conforme Figura \ref{fig:sistemaOperacionalComparacao}, cujas aplicações obtiveram acentuado crescimento nos últimos anos, como demonstram as Figuras \ref{fig:crescimentoAndroid} e \ref{fig:crescimentoIOS}.

\begin{figure}[h]
    \caption{\textit{Market Share} de sistemas operacionais em smartphones e tablets em janeiro de 2021.}
    \centering
    \includegraphics[width=1\textwidth]{img/sistema operacional.png}
    \fonte{\cite{multiplataformaSistemaOperacionalComparacao}}
    \label{fig:sistemaOperacionalComparacao}
\end{figure}

\begin{figure}[h]
    \caption{Crescimento de aplicações móveis no ambiente Android de dezembro de 2009 até dezembro de 2020.}
    \centering
    \includegraphics[width=.67\textwidth]{img/google play apps.png}
    \fonte{\cite{androidGraficoAndroid}}
    \label{fig:crescimentoAndroid}
\end{figure}


\begin{figure}[h]
    \caption{Crescimento de aplicações móveis no ambiente iOS de julho de 2008 até julho de 2020.}
    \centering
    \includegraphics[width=.67\textwidth]{img/app store apps.png}
    \fonte{\cite{iosGraficoIos}}
    \label{fig:crescimentoIOS}
\end{figure}

\newpage

O resultado prático desse aumento de aplicações móveis no mercado é a geração da demanda por desenvolvimento multiplataforma também crescente \cite{multiplataforma-IEEE}. Isso acontece porque aplicações multiplataforma reduzem o tempo e o custo de desenvolvimento e facilitam a manutenção da aplicação, uma vez que somente um código fonte é gerado e precisa ser mantido para diversas plataformas \cite{multiplataformaCross-plataform-diva}. Entretanto, existem alguns desafios quando se desenvolve uma aplicação multiplataforma \cite{multiplataforma-IEEE}, às vezes até chegando a inviabilizar a solução. São exemplos desses desafios:
\begin{itemize}
  \item \textit{Diferença de plataformas:} uma aplicação precisa executar em diferentes plataformas.
  \item \textit{Arquitetura de software:} uma arquitetura deve viabilizar ao máximo o compartilhamento de código reutilizável, considerando que uma pequena diferença entre plataformas pode gerar um grande impacto na arquitetura.
  \item \textit{Configuração de ferramentas:} é necessário utilizar ferramentas que estejam disponíveis nos diversos ambientes em que a aplicação será executada.
\end{itemize}


\section{Objetivo}
\label{sec:objetivo}

O presente trabalho visa contribuir para solucionar um dos desafios encontrados no desenvolvimento de aplicação multiplataforma, especificamente quanto à Arquitetura de Software. O objetivo deste trabalho é descrever uma arquitetura de software que viabiliza o desenvolvimento de aplicações móveis que compartilham código entre as plataformas Android e iOS.


\section{Relevância}
\label{sec:relevancia}

Os resultados do trabalho proposto podem facilitar o desenvolvimento de aplicações móveis multiplataforma, o que influencia em aplicações de alto padrão de qualidade menos custosas no seu desenvolvimento e de mais fácil manutenção.


\section{Resultados Esperados}
\label{sec:resuladosEsperados}

 Com a conclusão do trabalho, o resultado esperado é o desenvolvimento de uma arquitetura descrita e avaliada de forma que seja utilizada em trabalhos futuros que possuam o objetivo de desenvolver aplicações móveis multiplataforma.